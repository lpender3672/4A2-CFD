\documentclass{article}

\usepackage[utf8]{inputenc}

\usepackage{amsmath, bm}
\usepackage{graphicx}
\usepackage{amssymb}
\usepackage{float}
\usepackage{caption}
\usepackage{subcaption}
\usepackage{hyperref}
\usepackage{tikz}
\usepackage{layout}
\usepackage{booktabs}

\usepackage[margin=1in]{geometry}
\usepackage{listings}
\usepackage{xcolor}
\usepackage{color, colortbl}
\usepackage{textgreek}
\usepackage{mathrsfs}
\usepackage{savetrees}


\usetikzlibrary{calc}
\usetikzlibrary{angles,quotes} % for pic
\usetikzlibrary{patterns,snakes}
\usetikzlibrary{arrows}
\tikzset{>=latex} % for LaTeX arrow head

\setlength{\parskip}{\baselineskip}%
\setlength{\parindent}{0pt}%
\linespread{0.9}


\definecolor{codegreen}{rgb}{0,0.6,0}
\definecolor{codegray}{rgb}{0.5,0.5,0.5}
\definecolor{codepurple}{rgb}{0.58,0,0.82}
\definecolor{backcolour}{rgb}{0.95,0.95,0.92}

\lstdefinestyle{mystyle}{
    backgroundcolor=\color{backcolour},   
    commentstyle=\color{codegreen},
    keywordstyle=\color{magenta},
    numberstyle=\tiny\color{codegray},
    stringstyle=\color{codepurple},
    basicstyle=\ttfamily\footnotesize,
    breakatwhitespace=false,         
    breaklines=true,                 
    captionpos=b,                    
    keepspaces=true,                 
    numbers=left,                    
    numbersep=5pt,                  
    showspaces=false,                
    showstringspaces=false,
    showtabs=false,                  
    tabsize=2
}

\lstset{style=mystyle}



\begin{document}

\title{Computational Fluid Dynamics \\
    \large Final Report}
\author{lwp26}
\date{December 2024}
\maketitle 

\section{Software}
% monumental amount of work required here
\subsection{Modifications}
% interpolate
% write_settings python

Some corrections were made to the code supplied by the course.
The first correction was to the \texttt{interpolate} fortran function which only presented when compiling the program outside of debug mode and so was difficult to identify.
It was found that the "interp" function was returning unset memory values for the last point of the array. This is believed to be due to more relaxed precision in \texttt{si} which left it outside the normalised range of \texttt{si\_a}
The fix was to add a check to see if \texttt{si} is outside the input range, and if so return the boundary value.
A mistake in the python function \texttt{write\_settings} was also found where the gas constants were being written in the incorrect order.

The stopit file was removed and replaced with the variable \texttt{av\%crashed} which is set to true if \texttt{NaN} is detected, or if the user interrupts the program.

\subsection{Improvements}
% runge kutta
% deferred correction
% residual averaging
% spatially varying timestep
\subsubsection{Runge-Kutta}

\subsubsection{Deferred Correction}

\subsubsection{Residual Averaging}

\subsubsection{Spatially Varying Timestep}

% need to cherry pick commits
% and create 4 branches of individual improvements
% release 4 separate apps and produce comparison graphs
% discuss with reference to literature the improvement and implementation

\subsection{Extension}
% multigrid
% talk a bit about external flows
% comparison to boundary element methods e.g. SA1 spoilers its much less useful
% important cases evaluation turbinne_c represent same geometry with different
% mesh shapes make a large impact on the compuational effort accuracy curve.
% identify and make some recommonendations on mesh geometry


\subsection{Additional Cases}
% airfoil case comparison with SA1 or xfoil
% experiment with airfoils

% compare turbine theory with turbo course

\section{Results}

\subsection{Comparison of Improvements}

\subsection{Case Results}

\subsection{Effort vs Accuracy}

\section{Discussion}
% two pages that doesnt repeat interim

\section{Summary}

\end{document}